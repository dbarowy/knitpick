\documentclass[10pt]{article}

% Lines beginning with the percent sign are comments
% This file has been commented to help you understand more about LaTeX

% DO NOT EDIT THE LINES BETWEEN THE TWO LONG HORIZONTAL LINES

%---------------------------------------------------------------------------------------------------------

% Packages add extra functionality.
\usepackage{times,graphicx,epstopdf,fancyhdr,amsfonts,amsthm,amsmath,algorithm,algorithmic,xspace,hyperref}
\usepackage[left=1in,top=1in,right=1in,bottom=1in]{geometry}
\usepackage{sect sty}	%For centering section headings
\usepackage{enumerate}	%Allows more labeling options for enumerate environments 
\usepackage{epsfig}
\usepackage[space]{grffile}
\usepackage{booktabs}
\usepackage{forest}

% This will set LaTeX to look for figures in the same directory as the .tex file
\graphicspath{.} % The dot means current directory.

\pagestyle{fancy}

\lhead{Final Project}
\rhead{\today}
\lfoot{CSCI 334: Principles of Programming Languages}
\cfoot{\thepage}
\rfoot{Fall 2023}

% Some commands for changing header and footer format
\renewcommand{\headrulewidth}{0.4pt}
\renewcommand{\headwidth}{\textwidth}
\renewcommand{\footrulewidth}{0.4pt}

% These let you use common environments
\newtheorem{claim}{Claim}
\newtheorem{definition}{Definition}
\newtheorem{theorem}{Theorem}
\newtheorem{lemma}{Lemma}
\newtheorem{observation}{Observation}
\newtheorem{question}{Question}

\setlength{\parindent}{0cm}

%---------------------------------------------------------------------------------------------------------

% DON'T CHANGE ANYTHING ABOVE HERE

% Edit below as instructed

\begin{document}
  
\section*{Project Proposal}

Locke Meyer, Felix King

\subsection{Introduction}

\indent
KnitPick\#++ turns the tedious process of creating a knitting pattern, a set of instructions coupled with a chart that together 
describe the construction process of a specific garment, into a precise, painless, and intuitive programming experience. 
Ordinarily, creating a knitting pattern requires both some math skills as well as the graphic design savvy to draw a clear chart; 
with KnitPick\#++, these are no longer required. Furthermore, the various styles of knitting pattern that exist can be confusing 
to a knitter without a wide range of prior knitting experience; KnitPick\#++ establishes a powerful and expressive standard for 
knitting patterns that can be easily learned and read by anyone.

\bigskip

Many knitting patterns rely not only on repetition of single stitches, but also entire rows of stitches and of sets of rows. 
Also, since a secondary goal of knitting patterns is for a given pattern to produce the exact same garment every time, the output 
is deterministic. For these reasons, the problem of generating knitting patterns is particularly suited to a programming language-based 
solution, since programming languages offer easy repetition of pieces of information as well as precision.


\subsection{Design Principles}

One of our technical goals for KnitPick\#++ is for it to be relatively lightweight in terms of (e.g.) 
the number of basic stitches that are defined by the language. However, we also want to design KnitPick\#++ 
such that users can grow the language to fit their needs – for instance, by defining a less common stitch 
that serves a more niche purpose in the knitting process. Another one of our goals for KnitPick\#++ is to 
have it generate documents that look nice and are easy to comprehend. One way that this might inform the design 
of the language: rather than packing all possible information about a given pattern into a document for output, 
the user will be able to generate documents with varying amounts of detail based on their specifications.

\subsection{Examples}

\textbf{A.}
\begin{verbatim}
    pg Body = 
        row stock1 = s1 k+
        row stock2 = s1 p+
        repeat 75 stock1 stock2

    pg End = 
        "cast off"

    new pattern Scarf 50
        needles = 
            sp
            us 8
        yarn =
            alpaca
            4
        gauge = (8.0, 6.0)

    Body
    End
\end{verbatim}

\newpage
\textbf{B.}
\begin{verbatim}
    gr Checkerboard = 
        color orange = (255, 165, 0)
        color blue = (0, 0, 255)

        gridChunk checkerboard = 
            (orange 1 blue 1)+
            (blue 1 orange 1)+

        checkerboard 10

    pg Body = 
        row knit = k+
        repeat 50 knit BicolorPattern

    pg End =
        "cast off"

    new pattern BicolorTube 30
        needles = 
            ro
            8.0 mm
        yarn = 
            acrylic
            3
            blue, orange
        gauge = (10.0, 10.0)

    Checkerboard
    Body
    End
\end{verbatim}

\newpage
\textbf{C.}
\begin{verbatim}
    pg Ribbing =
        row rib = s1 k1 (p2 k2)+
        repeat 16 rib

    pg Body = 
        row stock = s1 k+
        repeat 75 stock

    pg End = 
        “cast off”

    new pattern LegWarmer 60
        needles = 
            ro
            us 6
        yarn =
            merino
            3
        gauge = (10.0, 8.0)

    Ribbing
    Body
    Ribbing
    End
\end{verbatim}

\subsection{Language Concepts}

The user will be thinking in terms of paragraphs consisting of text instructions and, optionally, grids defining more 
complex sequences of stitches. Each paragraph of text instructions will correspond to a part of the garment. Furthermore, 
each paragraph of text is made up of instructions, and these instructions are themselves made of up sequences including stitches, 
numbers, or other more specialized directives. For example, one paragraph of a sock knitting pattern might look like this.

\begin{verbatim}
Cuff:  
cast on 64

Row 1: knit 2 purl 2 to end of row.
Repeat Row 1 for 16 total rows.
\end{verbatim}

The primitives of this language are integers, floats, and strings. These primitives will all combine in various ways to create 
the paragraphs of text instructions in the final pattern. The floats will be used to construct colors, which will define rows 
of colors that can be built up to create the grids (if any) that appear in the final pattern. When a user is programming the 
paragraphs of text instructions, they will need to ensure that the numbers in the instructions they provide match the number of 
stitches they’ve instructed the knitter to use at that point in the knitting process. For instance, if the user attempts to 
generate the following instruction:

\begin{verbatim}
Cuff:  
cast on 64

Row 1: knit 2 purl 1 to end of row.
Repeat Row 1 for 16 total rows.
\end{verbatim}

The parser would recognize that 64 is not divisible by 3 and throw an error.

\subsection{Syntax}

Each program begins with the statement “new pattern” followed by a string, and then an integer. This string defines the name 
of the knitting pattern, and the integer defines the number of stitches to be casted on for this pattern. Underneath the 
“new pattern” statement, the user must define three fields – needles, yarn, and gauge, in that order – by typing the name of 
the field followed by ‘=’. The needles field defines the type of knitting needles to be used and their sizes, the yarn field 
defines the type of yarn to be used and its size, and the gauge defines the number of stitches per inch of fabric. All of the 
above constitutes the header of the program.

\bigskip

Once the header has been completed, the user will leave one line of whitespace and begin typing the body of the program. The 
body consists of paragraphs and grids. Most of the time, it will be convenient to create a variable for a paragraph or grid above 
the header.

\bigskip

A paragraph is defined by “pg” followed by a string, followed by a ‘=’. This string is the name of the paragraph. Underneath 
the “pg” statement, the user will define a string, a row, a repeat statement, a taper statement, or some combination of these. 
A row is defined by “row” followed by a string, followed by a ‘=’, followed by a sequence of stitches. A repeat is defined by 
“repeat” followed by an integer, followed by one or more rows. Here, the integer defines the number of times to repeat the rows 
in the statement. A taper is defined by “taper”, followed by a char, followed by an integer, followed by a ->, followed by 
another integer, followed by one or more rows. Here, the integers around the arrow define the starting and ending ‘widths’ of 
the taper pattern; this instruction is useful for creating rounded portions of garments (like the top of a beanie).

\bigskip

A grid is defined by “gr” followed by a string, followed by a ‘=’. This string is the name of the grid. Underneath the “gr” statement, 
the user will define one or more grid chunks and the number of times to repeat them in the overall grid. These grid chunks are defined 
by “gridChunk” followed by a string, followed by a ‘=’, followed by one or more grid rows. Grid rows are defined as one or more color 
sequences between parentheses, optionally followed by a ‘+’. Color sequences are a color followed by an integer. A color is defined by 
“color” followed by a string, followed by a ‘=’, followed by three comma-separated integers between parentheses.

\bigskip

\textbf{Minimal Formal Grammar}

For now, the start expression is stitchSeq.
\begin{verbatim}
    <stitchSeq> ::= <stitch><number>
                |   <plus><stitchSeq>
                |   <oParen><stitchSeq><stitchSeq><cParen>
    <stitch>    ::= k
                |   p
    <number>    ::= <d><number>
                |   <d>
    <d>         ::= 0 | 1 | 2 | 3 | 4 | 5 | 6 | 7 | 8 | 9
    <plus>      ::= +
    <oParen>    ::= (
    <cParen>    ::= )
\end{verbatim}

\subsection{Semantics}

\textbf{i.} 
The primitive values in KnitPick\#++ are integers, floats, and strings.

\bigskip

\textbf{ii.}
There are many possibilities, but the main way that values are combined is to build up paragraphs of text for the knitting 
pattern document. Generally, a user will use one or more stitches (each defined as a name, abbreviation, and weight) that 
are hard-coded into the language or define their own stitches to build up instructions. These instructions combine into 
paragraphs that describe how to knit the different parts of a given garment. 

\bigskip

Another way that values are combined is to build up grids. This process begins with a user defining one or more colors, 
which are tuples of three integers. Then, the user will define a row of colors by specifying a color and the number of 
times to repeat it in the row. After this, a user can define a chunk made up of many rows – basically, one “image” or 
portion of the overall pattern. Finally, these chunks are repeated an arbitrary number of times to create the final grid. 
The grid will be represented by an svg file that is imported into LaTeX to become part of the final pdf output.

\newpage
\textbf{iii.} 

\smallskip

Primitives
\begin{verbatim}
    Type integer = integer

    Type float = float

    Type string = string
\end{verbatim}

Combining Forms
\begin{verbatim}
    Type Stitch =  string * string * integer

    Type StitchSeq = 
    | Stitch * integer
    | StitchSeq * StitchSeq
    | StitchSeq * ‘+’

    Type Row = StitchSeq list

    Type Repeat = integer * Row list

    Type Taper = integer * integer * Row list

    Type Instruction = 
    | string
    | Row
    | Repeat
    | Taper
    | Instruction * string

    Type Paragraph = string * Instruction list

    Type Color = r : integer * g : integer * b : integer

    Type ColorSeq = Color * integer

    Type GridRow = 
        | ColorSeq list
        | ColorSeq list * ‘+’

    Type GridChunk = GridRow list

    Type Grid = string * (GridChunk * integer) list

    Type NeedleType = 
        | “sp”
        | “dp”
        | “ro”

    Type NeedleSize = 
        | “us” * integer
        | float * “mm”

    Type Needles = NeedleType * NeedleSize

    Type YarnColors = Color list

    Type Yarn = 
    | string * integer
    | string * integer * YarnColors

    Type Gauge = float * float

    Type Specifications = Needles * Yarn * Gauge

    Type Header = string * integer * Specifications

    Type Body = 
        | Body * Body
        | Paragraph
        | Grid

    Type Document = Header * Body
\end{verbatim}


\textbf{iv.}
Note: two of the following trees (B and C) were too wide to fit onto the page. In order to represent them, some portions of these trees, 
below the nodes marked "see below", have been split off and appear below the main tree, marked as "cont."

%first program example tree
\textbf{A.}
\begin{center}
    \begin{forest}
        [Document
            [Header
                ["Scarf"]
                [50]
                [Specifications
                    [Needles
                        [NType
                            ["sp"]
                        ]
                        [NSize
                            ["us"]
                            [8]
                        ]
                    ]
                    [Yarn
                        ["alpaca"]
                        [4]
                    ]
                    [Gauge
                        [8.0]
                        [6.0]
                    ]
                ]
            ]
            [Body
                [Paragraph
                    ["Body"]
                    [Instruction
                        [Repeat
                            [75]
                            [Row
                                [StitchSeq
                                    [Stitch
                                        ['s']
                                        ["slip"]
                                        [0]
                                    ]
                                    [1]
                                ]
                                [StitchSeq
                                    [Stitch
                                        ['k']
                                        ["knit"]
                                        [0]
                                    ]
                                    ['+']
                                ]
                            ]
                            [Row
                                [StitchSeq
                                    [Stitch
                                        ['s']
                                        ["slip"]
                                        [0]
                                    ]
                                    [1]
                                ]
                                [StitchSeq
                                    [Stitch
                                        ['p']
                                        ["purl"]
                                        [0]
                                    ]
                                    ['+']
                                ]
                            ]
                        ]
                    ]
                ]
                [Paragraph
                    ["End"]
                    [Instruction
                        ["cast off"]
                    ]
                ]
            ]
        ]
    \end{forest}
\end{center}

%second program example tree
\newpage
\textbf{B.}
\begin{center}
    \begin{forest}
        [Document
            [Header
                ["BicolorTube"]
                [30]
                [Specifications
                    [Needles
                        [NType
                            ["row"]
                        ]
                        [NSize
                            [8.0]
                            ["mm"]
                        ]
                    ]
                    [Yarn
                        ["acrylic"]
                        [3]
                        [YarnColors
                            ["blue"]
                            ["orange"]
                        ]
                    ]
                    [Gauge
                        [10.0]
                        [10.0]
                    ]
                ]
            ]
            [Body
                [Grid
                    [see below]
                ]
                [Paragraph
                    [see below]
                ]
                [Paragraph
                    ["End"]
                    [Instruction
                        ["cast off"]
                    ]
                ]
            ]
        ]
    \end{forest}

    \bigskip

    \begin{forest}
        [Grid (cont.)
            ["Checkerboard"]
            [GridChunk
                [GridRow
                    [ColorSeq
                        [orange]
                        [1]
                    ]
                    [ColorSeq
                        [blue]
                        [1]
                    ]
                    ['+']
                ]
                [GridRow
                    [ColorSeq
                        [blue]
                        [1]
                    ]
                    [ColorSeq
                        [orange]
                        [1]
                    ]
                ]
            ]
            [10]
        ]
    \end{forest}

    \bigskip

    \begin{forest}
        [Paragraph (cont.)
            ["Body"]
            [Instruction
                [Repeat
                    [50]
                    [Row
                        [StitchSeq
                            [Stitch
                                ['k']
                                ["knit"]
                                [0]
                            ]
                            ['+']
                        ]
                    ]
                ]
            ]
        ]
    \end{forest}
\end{center}

%third program example tree
\newpage
\textbf{C.}
\begin{center}
    \begin{forest}
        [Document
            [Header
                ["Legwarmer"]
                [60]
                [Specifications
                    [Needles
                        [NType
                            ["ro"]
                        ]
                        [NSize
                            ["us"]
                            [6]
                        ]
                    ]
                    [Yarn
                        ["merino"]
                        [3]
                    ]
                    [Gauge
                        [10.0]
                        [8.0]
                    ]
                ]
            ]
            [Body
                [Paragraph
                    [see below]
                ]
                [Paragraph
                    ["Body"]
                    [Instruction
                        [Repeat 
                            [75]
                            [Row 
                                [StitchSeq 
                                    [Stitch
                                        ['s']
                                        ["slip"]
                                        [0]
                                    ]
                                    [1]
                                ]
                                [StitchSeq 
                                    [Stitch
                                        ['k']
                                        ["knit"]
                                        [0]
                                    ]
                                    ['+']
                                ]
                            ]
                        ]
                    ]
                ]
                [Paragraph
                    ["End"]
                    [Instruction
                        ["cast off"]
                    ]
                ]
            ]
        ]
    \end{forest}

    \bigskip

    \begin{forest}
        [Paragraph (cont.)
            ["Ribbing"]
            [Instruction
                [Repeat
                    [16]
                    [Row
                        [StitchSeq
                            [Stitch
                                ['s']
                                ["slip"]
                                [0]
                            ]
                            [1]
                        ]
                        [StitchSeq
                            [Stitch
                                ['k']
                                ["knit"]
                                [0]
                            ]
                            [1]
                        ]
                        [StitchSeq
                            [StitchSeq
                                [Stitch
                                    ['p']
                                    ["purl"]
                                    [0]
                                ]
                                [2]
                            ]
                            [StitchSeq
                                [Stitch
                                    ['k']
                                    ["knit"]
                                    [0]
                                ]
                                [2]
                            ]
                            ['+']
                        ]
                    ]
                ]
            ]
        ]
    \end{forest}

\end{center}

\newpage
\textbf{v.}

\smallskip

\textbf{A.} 
Programs in KnitPick\#++ do not read any input. \medskip

\textbf{B.} 
The output of KnitPick\#++ is a pdf document containing a knitting pattern.
This result will be achieved by generating a LaTeX document (with optional svg output) and compiling it. \medskip

\textbf{C.} 
At the beginning of evaluation, KnitPick\#++ will set up a simple LaTeX document based on the 
Header specified by the programmer. This includes several settings that the programmer will not choose, 
like page style, text size, etc. Since every pattern requires "casting on," that is, creating the first 
stitches, a "Beginning" section will be generated with the simple instruction to cast on the number of 
stitches specified in the Header. \medskip

Below is a tree representing the point in the evaluation of example AST (A.) where only the Header branch of the 
AST has been evaluated and the LaTeX code KnitPick\#++ would generate. 

%first evaluation tree
\begin{center}
    \begin{forest}
        [Document
            [LaTeX code (see below)]
            [Body
                [Paragraph
                    ["Body"]
                    [Instruction
                        [Repeat
                            [75]
                            [Row
                                [StitchSeq
                                    [Stitch
                                        ['s']
                                        ["slip"]
                                        [0]
                                    ]
                                    [1]
                                ]
                                [StitchSeq
                                    [Stitch
                                        ['k']
                                        ["knit"]
                                        [0]
                                    ]
                                    ['+']
                                ]
                            ]
                            [Row
                                [StitchSeq
                                    [Stitch
                                        ['s']
                                        ["slip"]
                                        [0]
                                    ]
                                    [1]
                                ]
                                [StitchSeq
                                    [Stitch
                                        ['p']
                                        ["purl"]
                                        [0]
                                    ]
                                    ['+']
                                ]
                            ]
                        ]
                    ]
                ]
                [Paragraph
                    ["End"]
                    [Instruction
                        ["cast off"]
                    ]
                ]
            ]
        ]
    \end{forest}
\end{center}

%first eval code (header evaluated)
\begin{verbatim}
    \documentclass[10pt]{article}

    \usepackage{times,graphicx,fancyhdr,amsfonts,xspace,hyperref}
    \usepackage[left=1in,top=1in,right=1in,bottom=1in]{geometry}
    \usepackage{sect sty}	%For centering section headings
    \usepackage{enumerate}	%Allows more labeling options for enumerate environments 
    \usepackage[space]{grffile}

    \graphicspath{.} 

    \pagestyle{fancy}

    \renewcommand{\headrulewidth}{0.4pt}
    \renewcommand{\headwidth}{\textwidth}
    \renewcommand{\footrulewidth}{0.4pt}

    \setlength{\parindent}{0cm}

    \title{Scarf}
    \date{}

    \begin{document}

    \maketitle

    \section*{Project Specifications}

    \textbf{Needles}: single pointed size US 8

    \textbf{Yarn}: medium size alpaca

    \textbf{Gauge}: 8.0 stitches and 6.0 rows per square inch

    \section*{Beginning}

    Cast on 50 stitches.

    \end{document}

\end{verbatim}

\medskip
Next, the rest of the tree would be evaluated to put together paragraphs containing instructions pertaining
to the rest of the knitting project. Stitch sequences will be put together into short strings. For example, 
the bottom left stitch sequence will be made into the following string: "slip 1 stitch". The "+" in a 
stitch sequence indicates that the knitter should repeat this stitch sequence until the end of the row, so
the next stitch sequence (following post-order traversal) will be turned into "knit to end of row." These
last two stitch sequences will be combined to describe a single row of knitting. Below is the next tree and
just the code generated to represent the row just discussed. \medskip

%second evaluation tree
\begin{center}
    \begin{forest}
        [Document
            [LaTeX code]
            [Body
                [Paragraph
                    ["Body"]
                    [Instruction
                        [Repeat
                            [75]
                            [LaTex code (see below)]
                            [Row
                                [StitchSeq
                                    [Stitch
                                        ['s']
                                        ["slip"]
                                        [0]
                                    ]
                                    [1]
                                ]
                                [StitchSeq
                                    [Stitch
                                        ['p']
                                        ["purl"]
                                        [0]
                                    ]
                                    ['+']
                                ]
                            ]
                        ]
                    ]   
                ]
                [Paragraph
                    ["End"]
                    [Instruction
                        ["cast off"]
                    ]
                ]
            ]
        ]
    \end{forest}
\end{center}

% row 1 code
\begin{verbatim}
    \textbf{Row 1}: slip 1 stitch, knit to end of row.
\end{verbatim}

\newpage
The same will be done for the other row in the "Body" paragraph of the pattern, and these two rows
will be referred to in the "repeat" instruction. Evaluating the repeat will result in another string:
"Repeat rows 1 and 2 for 75 total rows." These two rows and the repeat instruction will be displayed
in the Body paragraph using the following LaTeX:

% third evaluation tree
\begin{center}
    \begin{forest}
        [Document
            [LaTeX code]
            [Body
                [LaTeX code (see below)]
                [Paragraph
                    ["End"]
                    [Instruction
                        ["cast off"]
                    ]
                ]
            ]
        ]
    \end{forest}
\end{center}

% Body section code
\begin{verbatim}
    \section*{Body}

    \textbf{Row 1}: slip 1 stitch, knit to end of row.

    \textbf{Row 2}: slip 1 stitch, purl to end of row.

    Repeat rows 1 and 2 for 75 total rows. 
\end{verbatim}

\medskip
The same will be done for the "End" paragraph, which only contains one brief instruction. 
The LaTeX generated for these two paragraphs can be put together in sequence. Finally, the 
Header and the Body of the document can be combined by placing the code for the Body between 
just before the end\{document\} line in the LaTeX document. The full concept LaTeX document 
(scarf.tex) is in this directory, as is the pdf generated from it (scarf.pdf). 

% DO NOT DELETE ANYTHING BELOW THIS LINE
\end{document}
