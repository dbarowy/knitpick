\documentclass[10pt]{article}

% Lines beginning with the percent sign are comments
% This file has been commented to help you understand more about LaTeX

% DO NOT EDIT THE LINES BETWEEN THE TWO LONG HORIZONTAL LINES

%---------------------------------------------------------------------------------------------------------

% Packages add extra functionality.
\usepackage{times,graphicx,epstopdf,fancyhdr,amsfonts,amsthm,amsmath,algorithm,algorithmic,xspace,hyperref}
\usepackage[left=1in,top=1in,right=1in,bottom=1in]{geometry}
\usepackage{sect sty}	%For centering section headings
\usepackage{enumerate}	%Allows more labeling options for enumerate environments 
\usepackage{epsfig}
\usepackage[space]{grffile}
\usepackage{booktabs}
\usepackage{forest}
\usepackage{array}


% This will set LaTeX to look for figures in the same directory as the .tex file
\graphicspath{.} % The dot means current directory.

\pagestyle{fancy}

\lhead{Final Project}
\rhead{\today}
\lfoot{CSCI 334: Principles of Programming Languages}
\cfoot{\thepage}
\rfoot{Fall 2023}

% Some commands for changing header and footer format
\renewcommand{\headrulewidth}{0.4pt}
\renewcommand{\headwidth}{\textwidth}
\renewcommand{\footrulewidth}{0.4pt}

% These let you use common environments
\newtheorem{claim}{Claim}
\newtheorem{definition}{Definition}
\newtheorem{theorem}{Theorem}
\newtheorem{lemma}{Lemma}
\newtheorem{observation}{Observation}
\newtheorem{question}{Question}

\setlength{\parindent}{0cm}

%---------------------------------------------------------------------------------------------------------

% DON'T CHANGE ANYTHING ABOVE HERE

% Edit below as instructed

\begin{document}
  
\section*{KnitPick\#++ Specification}

Locke Meyer, Felix King

\subsection{Introduction}

\indent
KnitPick\#++ turns the tedious process of creating a knitting pattern, a set of instructions that 
describe the construction process of a specific garment, into a precise, painless, and intuitive programming experience. 
Ordinarily, creating a knitting pattern requires the graphic design savvy to organise the 
document cleanly and clearly; with KnitPick\#++, this are no longer required. Furthermore, the various styles of knitting pattern that 
exist can be confusing to a knitter without a wide range of prior knitting experience; KnitPick\#++ establishes a powerful 
and expressive standard for knitting patterns that can be easily learned and read by anyone.

\bigskip

Many knitting patterns rely not only on repetition of single stitches, but also entire rows of stitches and of sets of rows. 
Also, since a secondary goal of knitting patterns is for a given pattern to produce the exact same garment every time, the output 
is deterministic. For these reasons, the problem of generating knitting patterns is particularly suited to a programming language-based 
solution, since programming languages offer easy repetition of pieces of information as well as precision.

\subsection{Design Principles}

One of our technical goals for KnitPick\#++ is for it to be relatively lightweight in terms of (e.g.) 
the number of basic stitches that are defined by the language. However, we also want to design KnitPick\#++ 
such that users can define new stitch types to fit their needs – for instance, by 
defining a less common stitch that serves a more niche purpose in the knitting process, like the moss stitch. Another one of our 
goals for KnitPick\#++ is to have it generate documents that look nice and are easy to comprehend. An aesthetic goal 
is for KnitPick\#++ code to be concise and readable; for this reason, many keywords in our syntax are abbreviations.

\subsection{Examples}

Each of the following examples can be executed from within the examples folder 
by typing the following into the terminal: 
"dotnet run --project ../../code/kpplibrary example-1.kpp example-1". Here, "example-1.kpp" is the name of the file containing KnitPick\#++ code 
and "example-1" is the name of the pdf file that will be generated. 

KnitPick\#++ will automatically create a pdf of your knitting pattern.

\textbf{A.}
\begin{verbatim}
header "Potholder"
    needle sp "us" 8
    gauge (5.0, 5.0)
    yarn "Cotton" 6

pg "Begin"
    "Cast on 35 stitches.";

pg "Body"
    repeat +k1, 35;

pg "End"
    "Cast off and sew in loose end.";
\end{verbatim}

\newpage
\textbf{B.}
\begin{verbatim}
let s = "slip"

header "Chunky Scarf"
    needle sp "us" 8
    gauge (4.0,4.0) 
    yarn "Mohair-merino" 5


pg "Beginning Ribbing"
    "Cast on 64";
    repeat +(k1p1), 12;

pg "Body"
    repeat s1, +k1, s1, +p1, 240;

pg "End Ribbing"
    repeat +(k1p1), 12;
    "Cast off and sew in ends";
\end{verbatim}

\textbf{C.}
\begin{verbatim}
let tog = "knit two together"

header "Beanie"
    needle ci "us" 5
    gauge (8.0, 8.0)
    yarn "merino" 3

pg "Ribbing"   
    "Cast on 88 stitches.";
    "Join stitches in the round.";
    repeat +(k2p2), 24;

pg "Body"
    repeat +(k4p4), 4;
    repeat +(p4k4), 4;
    "continue in this manner until garment measures 8.5 inches";

pg "Taper"
    repeat +k1, +(k7tog1), 10;

pg "End"
    "Cut yarn to about 1 foot long";
    "Using an embroidery needle, thread together last stitches and tie a knot";
    "Sew in the ends";
\end{verbatim}
\newpage

\subsection{Language Concepts}

The user will be thinking in terms of \textbf{paragraphs} consisting of text instructions. Each paragraph of text
will correspond to a part of the garment -- for instance, the ribbing of a scarf or the cuff of a sweater. 
Furthermore, each paragraph of text is made up of \textbf{instructions}, and these instructions are themselves made up 
of \textbf{sequences} including \textbf{stitches}, numbers, or other more specialized directives. For example, one paragraph of a sock 
knitting pattern might look like this.

\begin{verbatim}
Cuff
Step 1. cast on 64.
Step 2. 
    (a) knit 2, purl 2 to end of row.
Repeat for a total of 16 times.
\end{verbatim}

The primitives of this language are integers, floats, and strings. These primitives will all combine in various ways to create 
the paragraphs of text instructions in the final pattern. Integers are used to represent the number of repetitions of a stitch and
the number of times to repeat an instruction, among other things. Strings represent the names of stitches and any custom
instruction the user wants to include. Floats appear in the header of a document and provide precise information about the "gauge" of the 
pattern and the size of the yarn to be used.

\newpage

\subsection{Formal Syntax}
The start expression is \verb|<| program \verb|>|.
\begin{verbatim}
    <program>     ::= (<assignment>)*<document>
    <assignment>  ::= "let "<variable>" ="<pad><string><pad>
    <variable>    ::= (<letter>)+
    <document>    ::= <pad><header><pad>(<pad><paragraph><pad>)+
    <header>      ::= "header "<pad><string><pad><needle><pad><gauge><pad><yarn>
    <needle>      ::= "needle "<pad>"sp"<pad><string><pad><integer>
                  |   "needle "<pad>"dp"<pad><string><pad><integer>
                  |   "needle "<pad>"ci"<pad><string><pad><integer>
    <gauge>       ::= "gauge "<oParen><float><pad><comma><pad><float><cParen> 
    <yarn>        ::= "yarn "<pad><string><pad><integer>
    <paragraph>   ::= "pg "<string>(<pad><instruction><pad>)+
    <instruction> ::= <string>
                  |   <row>
    <row>         ::= (<stitchSeq><pad>)+
    <stitchSeq>   ::= <stitch><integer>
                  |   <plus><stitchSeq>
                  |   <oParen><stitchSeq><stitchSeq><cParen>
    <stitch>      ::= 'k'
                  |   'p'
                  |   <variable>
    <pad>         ::= (<ws>)*
    <string>      ::= <quote>(<character>)+<quote>
    <integer>     ::= (<d>)+
    <float>       ::= (<d>)+<dot>(<d>)+
    <d>           ::= '0' | '1' | ... | '8' | '9'
    <period>      ::= '.'
    <plus>        ::= '+'
    <oParen>      ::= '('
    <cParen>      ::= ')'
    <comma>       ::= ','
    <ws>          ::= ' ' | '\n' | '\t' | "\rn"
    <quote>       ::= '"'
    <letter>      ::= 'A' | ... | 'Z' | ... | 'a' | ... | 'z' |
    <character>   ::= ' ' | '!' | ... | 'a' | ... | 'z' | ... | '}' | '~'
\end{verbatim}

\newpage

\subsection{Semantics}

\begin{tabular}{c|m{3.5cm}|m{3.5cm}|c|m{4cm}}
    
    Syntax & Abstract Syntax & Type Syntax & Prec./assoc. & Meaning \\
    \hline
    \textit{v} & \texttt{Var of String} & \texttt{string} & n/a & \textit{v} represents a variable, which can store a stitch. Every use of the variable gets replaced with the stitch it represents. \\ 
    \hline 
    "let "\textit{var}" = "\textit{st} & \texttt{Assignment of Var * String} & \texttt{Assignment} & n/a & An Assignment associates a var with a string for later use. \\
    \hline
    \textit{n} & \texttt{Int of int} & \texttt{int} & n/a & \textit{n} is a primitive. We represent integers using the 32-bit F\# integer data type (int32). \\
    \hline
    "\textit{str}" & \texttt{String of string} & \texttt{string} & n/a & \textit{str} is a primitive. We represent strings using the built-in F\# string type. \\
    \hline
    'k' or 'p' & \texttt{Stitch of String * Int} & \texttt{string * int} & n/a & 'k' or 'p' represents either a knit or a purl stitch. We construct a Stitch object by keeping a (small) dictionary of stitches and their abbreviations. \\
    \hline
    \textit{sn} & \texttt{StitchRep of Stitch * Int} & \texttt{(string * int) * int} & n/a & \textit{sn} is a combining form that communicates a particular kind of stitch (knit or purl) and the number of times it should be performed. \\
    \hline
    +\textit{S} & \texttt{StitchSeqToEnd of StitchSeq} & \texttt{StitchSeq} & n/a & +\textit{S} is a kind of StitchSeq that indicates that the StitchSeq \textit{S} should be repeated until the end of the row. \\
    \hline
    (\textit{SS}) & \texttt{TwoStitchSeq of StitchSeq * StitchSeq} & \texttt{StitchSeq * StitchSeq} & n/a & (\textit{SS}) is a combining form that allows StitchSeqs to be chained for more complexity. \\
    \hline 
    \textit{R} & \texttt{Row of StitchSeq list} & \texttt{StitchSeq list} & n/a & A row combines multiple stitch sequences that, together, make up one row of knitting. \\
    \hline
    \textit{I} & \texttt{Instruction} & \texttt{Instruction} & n/a & \textit{I} is an instruction, either a row of knitting or other direction (for example, "cast on.") \\ 
    \hline

    
\end{tabular}


\begin{tabular}{m{3cm}|m{3.5cm}|m{3.5cm}|c|m{4cm}}
    Syntax & Abstract Syntax & Type Syntax & Prec./assoc. & Meaning \\
    \hline
    "pg "\textit{name} \textit{inst} & \texttt{Paragraph of String * Instruction list} & \texttt{string * Instruction list} & n/a & The string "pg " followed by a string and a list of instructions defines a paragraph. The string is the name of the paragraph and the list of instructions describes how that part of the garment should be knit. \\
    \hline
    "needle " \textit{ty sys size} & \texttt{Needle of String * String * Int} & \texttt{string * string * int} & n/a & The string "needle" followed by two strings and an integer represents the needles needed for the knitting pattern. \textit{ty} represents the type of needle, \textit{sys} represents the measurement system (either US or mm), and \textit{size} is the size in that system. \\ 
    \hline
    "gauge " (\textit{stitches, rows}) & \texttt{Gauge of Float * Float} & \texttt{float * float} & n/a & The string "gauge " followed by a tuple of floats in parentheses communicates the ratios of stitches to horizontal inches and rows to vertical inches that the pattern designer used. \\
    \hline
    "yarn " \textit{"type" size} & \texttt{Yarn of String * int} & \texttt{string * int} & n/a & "yarn " followed by a string and and integer represents the type / brand and the size of the yarn used in the pattern \\
    \hline
    "header " \textit{name needle gauge yarn} & \texttt{Header of String * Needle * Gauge * Yarn} & \texttt{Header} & n/a & The header data type is required in each pattern and contains the information about the needles, gauge, and yarn. \\
    \hline
    \textit{header pgs} & \texttt{Document of Header * Paragraph list} & \texttt{Document} & n/a & A Document consists of a Header followed by a list of paragraphs. \\
    \hline
    \textit{assigns doc} & \texttt{Program of Assignment list * Document} & \texttt{Assignment} & n/a & A program wraps a set of assignments and the Document datatype in that order. \\ 
    \hline

 
\end{tabular}


\subsection{Remaining Work}

In our previous draft, we had mentioned wanting to implement a Taper type -- this actually turned out to be redundant due to the Repeat type. 
We also mentioned that we wanted to implement variables. And we did! HOWEVER, due to the design of our whole abstract syntax tree, we realized that we wouldn't be able to implement a variable that could represent
any arbitrary data type. This was a valuable learning experience! We did end up implementing variables anyway; it's just that they
can only represent stitches. 

In the future, it would be nice to add a way for the user to create a colored grid pattern as a visual aid to the knitting instructions. 
This was something we had originally thought about doing. However, we realized early on in the process of designing KnitPick\#++ 
that we just wouldn't end up having time to implement it, but it could be cool to come to and work on at some point.

% DO NOT DELETE ANYTHING BELOW THIS LINE
\end{document}
